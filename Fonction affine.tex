\documentclass{article}
\usepackage{titling} % Package pour personnaliser les titres
\usepackage{amsmath, amssymb} % Pour des fonctionnalités mathématiques améliorées
\usepackage{titlesec}
\usepackage{tikz}
\usepackage{array}
\usepackage{pgfplots}
\pgfplotsset{compat=1.17}
\usepackage{lipsum} % Pour générer du texte factice, optionnel
\usepackage[utf8]{inputenc} % Encodage UTF-8
\usepackage[a4paper,margin=2.5cm]{geometry} % Marges de 2.5 cm
\usepackage{fancyhdr} % Pour personnaliser les en-têtes et pieds de page
\usepackage[T1]{fontenc}
\usepackage[french]{babel}
\usepackage{lastpage} % Permet de récupérer le numéro de la dernière page
\usepackage{tabularx}
\usepackage{tcolorbox}
\usepackage{multicol}
\usepackage{enumitem}
\usepackage{setspace} % permet de gérer l'interligne
\usepackage{parskip} % Espacement automatique des paragraphes

% Configuration des en-têtes et pieds de
% Configuration du package fancyhdr
\pagestyle{fancy}
\fancyhf{} % Efface les styles par défaut
\fancyhead[L]{\textbf{Cours - La fonction affine}} % En-tête à gauche
\fancyhead[R]{\textbf{Décembre 2024}}
\fancyfoot[R]{Page \thepage\ sur \pageref{LastPage}} % Numéro de page centré en bas
\fancyfoot[L]{} % Pied de page à gauche

% Suppression de la ligne horizontale par défaut dans l'en-tête
\renewcommand{\headrulewidth}{0.4pt} % Ligne fine sous l'en-tête
\renewcommand{\footrulewidth}{0.4pt} % Ligne fine au-dessus du pied de page

\begin{document}

% Définition
\section{\textbf{Définition}}

\begin{itemize}
    \item a et b désignent deux nombres réels fixés. Une fonction affine f est une fonction définie sur \(\mathbb{R}\) par la relation \(f(x)=ax+b\).
    \item Si a = 0 : f est une \textbf{fonction constante} \(f(x)=b\)
    \item Si b = 0 : f est une \textbf{fonction linéaire} de coefficient a \(f(x)=ax\)
    \vspace*{1cm}
    \item Toutes les fonctions linéaires sont des fonctions affines.
    \item Toutes les fonctions constantes sont des fonctions affines.
\end{itemize}

\section{\textbf{Images et antécédents}}
\begin{itemize}
    \item On dit que le nombre réel \(f(x)\) est l'\textbf{image} du nombre réel x par la fonction f ; 
    \item On dit que le nombre réel x est l'\textbf{antécédent} du nombre réel \(f(x)\) par la fonction f.
\end{itemize}

Exemple : On considère la fonction f définie sur \(\mathbb{R}\) telle que \(f(x)=-3x+1\)
\begin{itemize}
    \item L'image de 2 est \(f(2)=-3 \times 2 + 1 = -6+1 = -5\)
    \item Pour calculer un antécédant, il faut résoudre une équation :  \(-3x+1=2\). On dit que -5 est l'antécédent de 2 est \(-5\)
\end{itemize}

\section{\textbf{Représentation graphique}}

\begin{center}
    \begin{tikzpicture}
        \begin{axis}[
            axis lines = middle,
            xmin=-5, xmax=5, % Plage de x
            ymin=-5, ymax=5, % Plage de y
            xlabel = \(x\),
            ylabel = \(f(x)\),
            grid = both,
            width=10cm, % Largeur du graphique
            height=10cm, % Hauteur du graphique
            domain=-5:5, % Plage de la fonction
            unit vector ratio = 1:1,
            xtick={-5,-4,-3,-2,-1,0,1,2,3,4,5}, % Positions des graduations principales sur l'axe x
            ytick={-5,-4,-3,-2,-1,0,1,2,3,4,5},
            major grid style={line width=0.8pt, draw=black!70}, % Grille principale plus épaisse et foncée
            minor grid style={line width=0.5pt, draw=black!50}, % Grille secondaire plus fine et foncée
            samples=100
            ]
            \addplot[thick, blue] {-3*x +1};
        \end{axis}
    \end{tikzpicture}
\end{center}

On dit que cette droite a pour équation \(y = ax + b\)
\begin{itemize}
    \item a est le \textbf{coefficient directeur} de la droite 
    \item b est l'\textbf{ordonnée à l'origine} de la droite (où la droite coupe l'axe des ordonnée à x=0)
\end{itemize}

\newpage
\section{\textbf{Méthode pour retrouver le coefficient directeur (a)}}

\begin{enumerate}
    \item \textbf{Méthode algébrique :} 
    \[
    \text{Soit deux points \( A(x_A, y_A) \) et \( B(x_B, y_B) \) de coordonnées, le coefficient directeur \( a \) est donné par :}
    \]
    \[
    a = \frac{y_B - y_A}{x_B - x_A} = \frac{f(x_B) - f(x_A)}{x_B - x_A}
    \]

    \item \textbf{Méthode graphique :} 
    \begin{itemize}
        \item On choisit deux points sur le plan
        \item puis on calcule les déplacements pour aller du premier au deuxième point (haut/bas et gauche/droite)
        \item Par exemple, sur le plan ci-dessus, on descend de 3 unités et on se déplace à droite de 1 unité. Le coefficient directeur de cette fonction est alors :
        \[
        a = \frac{-3}{1} = -3
        \]
    \end{itemize}
\end{enumerate}





\section{\textbf{Variation et signe d'une fonction affine}}

Le sens de variation d'une fonction affine dépend du signe du coefficient directeur (a) \\
Ce coefficient directeur représente la "pente" de la droite représentative de f.
\begin{itemize}
    \item si \(a < 0\) alors la fonction est décroissante.
    \item si \(a = 0\) alors la fonction est constante.
    \item si \(a > 0\) alors la fonction est croissante.
\end{itemize}

\begin{itemize}
    \item Si \(a > 0\)
    \begin{enumerate}
        \item Son tableau de signe : 
        \[
        \renewcommand{\arraystretch}{2}
        \begin{array}{|c|ccccc|}
        \hline
        x & -\infty & & \frac{-b}{a} & & +\infty \\ \hline
        \text{signe de} \, f(x) & & - & 0 & + & \\ \hline
        \end{array}
        \]
        \vspace*{1cm}
        \item Son tableau de variation : 
        \[
        \renewcommand{\arraystretch}{2}
        \begin{array}{|c|ccccc|}
        \hline
        x & -\infty & & \frac{-b}{a} & & +\infty \\ \hline
        \text{Variation de} \, f(x) &  & \nearrow & 0 & \nearrow &  \\ \hline
        \end{array}
        \]
    \end{enumerate}
    
    \item Si \(a < 0\)
    \begin{enumerate}
        \item Son tableau de signe : 
        \[
        \renewcommand{\arraystretch}{2}
        \begin{array}{|c|ccccc|}
        \hline
        x & -\infty & & \frac{-b}{a} & & +\infty \\ \hline
        \text{signe de} \, f(x) & & + & 0 & - & \\ \hline
        \end{array}
        \]
        \vspace*{1cm}
        \item Son tableau de variation : 
        \[
        \renewcommand{\arraystretch}{2}
        \begin{array}{|c|ccccc|}
        \hline
        x & -\infty & & \frac{-b}{a} & & +\infty \\ \hline
        \text{Variation de} \, f(x) &  & \searrow & 0 & \searrow &  \\ \hline
        \end{array}
        \]
    \end{enumerate}
\end{itemize}



\end{document}

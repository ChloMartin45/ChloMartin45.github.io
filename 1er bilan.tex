\documentclass{article}
\usepackage{titling} % Package pour personnaliser les titres
\usepackage{amsmath, amssymb} % Pour des fonctionnalités mathématiques améliorées
\usepackage{titlesec}
\usepackage{tikz}
\usepackage{array}
\usepackage{pgfplots}
\pgfplotsset{compat=1.17}
\usepackage{lipsum} % Pour générer du texte factice, optionnel
\usepackage[utf8]{inputenc} % Encodage UTF-8
\usepackage[a4paper,margin=2.5cm]{geometry} % Marges de 2.5 cm
\usepackage{fancyhdr} % Pour personnaliser les en-têtes et pieds de page
\usepackage[T1]{fontenc}
\usepackage[french]{babel}
\usepackage{lastpage} % Permet de récupérer le numéro de la dernière page
\usepackage{tabularx}
\usepackage{tcolorbox}
\usepackage{multicol}
\usepackage{enumitem}
\usepackage{setspace} % permet de gérer l'interligne
\usepackage{parskip} % Espacement automatique des paragraphes

% Configuration des en-têtes et pieds de
% Configuration du package fancyhdr
\pagestyle{fancy}
\fancyhf{} % Efface les styles par défaut
\fancyhead[L]{\textbf{1^{er} bilan}} % En-tête à gauche
\fancyhead[R]{\textbf{20/12/2024}}
\fancyfoot[R]{Page \thepage\ sur \pageref{LastPage}} % Numéro de page centré en bas
\fancyfoot[L]{} % Pied de page à gauche

% Suppression de la ligne horizontale par défaut dans l'en-tête
\renewcommand{\headrulewidth}{0.4pt} % Ligne fine sous l'en-tête
\renewcommand{\footrulewidth}{0.4pt} % Ligne fine au-dessus du pied de page

% Configuration des tailles de sections et sous-sections
\titleformat{\section}{\normalfont\large\bfseries}{\thesection}{1em}{}
\titleformat{\subsection}{\normalfont\normalsize\bfseries}{\thesubsection}{1em}{}
\titleformat{\subsubsection}{\normalfont\small\bfseries}{\thesubsubsection}{1em}{}

\begin{document}

% Equation
\section{\textbf{Les équations}} 

\begin{enumerate}
    \item Résoudre ces équations.
    \begin{enumerate}
        \item \(3x+15 = 0\) \vspace{3cm}
        \item \(8x + 6 = 19x-3\)\vspace{3cm}
        \item \((2x+4)(6x-2)=0\) \vspace{3cm}
    \end{enumerate}
    \item Résoudre ces inéquations. 
    \begin{enumerate}
        \item \(x+5 \leq 4(x-1)+7\)\vspace{4cm}
        \item \(-4x + 2 \geq -9\) \vspace{3cm}     
    \end{enumerate}
\end{enumerate}

\newpage
% Fonctions
\section{\textbf{Les fonctions}} 

\begin{center}
    \begin{tikzpicture}
        \begin{axis}[
            axis lines = middle,
            xmin=-4, xmax=4, % Plage de x
            ymin=-2, ymax=6, % Plage de y
            xlabel = \(x\),
            ylabel = \(y\),
            grid = both,
            width=7cm, % Largeur du graphique
            height=7cm, % Hauteur du graphique
            domain=-4:4, % Plage de la fonction
            unit vector ratio = 1:1,
            major grid style={line width=0.8pt, draw=black!70}, % Grille principale plus épaisse et foncée
            minor grid style={line width=0.5pt, draw=black!50}, % Grille secondaire plus fine et foncée
            samples=100
            ]
        \addplot[thick, blue] {0.25*x^2 -1}; 
        \end{axis}
    \end{tikzpicture}
\end{center}
    
\begin{enumerate}[label=\textbf{Question \arabic*.}]
    \item Quelle est l'image de \(0\) par la fonction \(h\) ?
    \vspace{0.5cm} 
    \begin{itemize}[label=]
        \item[A.] \( h(0) = 6 \) \vspace{0.5cm} 
        \item[B.] \( h(0) = 0 \) \vspace{0.5cm} 
        \item[C.] \( h(0) = -1 \) \vspace{0.5cm} 
        \item[D.] Autre réponse
    \end{itemize}
    \vspace{0.5cm} 
    
    \item Quels nombres ont pour image \(0\) par la fonction \(h\) ?
    \vspace{0.5cm}
    \begin{itemize}[label=]
        \item[A.] \( x = -2 \) et \( x = 2 \) \vspace{0.5cm} 
        \item[B.] \( x = 0 \) et \( x = 1 \) \vspace{0.5cm} 
        \item[C.] \( x = -1 \) et \( x = 0 \) \vspace{0.5cm} 
        \item[D.] Autre réponse
    \end{itemize}
    \vspace{0.5cm} 
    
    \item Donner une valeur approchée de l'image de \(-3\) par la fonction \(h\).
    \vspace{0.5cm}
    \begin{itemize}[label=]
        \item[A.] \( h(-3) \approx 1.25 \) \vspace{0.5cm} 
        \item[B.] \( h(-3) \approx -1.25 \) \vspace{0.5cm} 
        \item[C.] \( h(-3) \approx -4 \) \vspace{0.5cm} 
        \item[D.] Autre réponse
    \end{itemize}
\end{enumerate}

% Moyenne - Quartile - Étendue
\newpage
\section{\textbf{Moyenne - Quartile - Étendue}} 

\text{On a rangé dans un tableau les tailles d'un groupe de personnes.}

\begin{table}[h!]
    \centering
    \begin{tabular}{|c|c|c|c|c|c|c|c|} % 8 colonnes
        \hline
        \textbf{Taille (cm)} & 130 & 145 & 155 & 160 & 170 & 175 & 185 \\ \hline
        \textbf{Effectif}     & 3  & 5  & 6 & 2 & 1  & 5 & 2  \\ \hline
    \end{tabular}
    \caption{Tableau des tailles et effectifs}
\end{table}

\begin{enumerate}[label=\textbf{Question \arabic*.}]
    \item Calculer la taille moyenne de ce groupe de personnes ?
    \vspace{0.5cm} 
    \begin{itemize}[label=]
        \item[A.] \( m=160 cm \) \vspace{0.5cm} 
        \item[B.] \( m=158,5 cm\) \vspace{0.5cm} 
        \item[C.] \( m=157,5 cm \) \vspace{0.5cm} 
        \item[D.] Autre réponse
    \end{itemize}
    \vspace{0.5cm} 
    \item Calculer le premier quartile de cette série.
    \vspace{0.5cm}
    \begin{itemize}[label=]
        \item[A.] \( Q_1 = 145 cm \) \vspace{0.5cm} 
        \item[B.] \( Q_1 = 155 cm\) \vspace{0.5cm} 
        \item[C.] \( Q_1 = 160 cm \) \vspace{0.5cm} 
        \item[D.] Autre réponse
    \end{itemize}
    \vspace{0.5cm} 
        \item Calculer l'étendue de cette série.
    \vspace{0.5cm}
    \begin{itemize}[label=]
        \item[A.] \( 55 cm \) \vspace{0.5cm} 
        \item[B.] \( 155 cm \) \vspace{0.5cm} 
        \item[C.] \( 315 cm \) \vspace{0.5cm} 
        \item[D.] Autre réponse
    \end{itemize}
\end{enumerate}

% Fraction

\newpage
\section{\textbf{Les fractions}} 
\begin{enumerate}
    \item Multiplier ces fractions et les simplifier si besoin.
    \begin{multicols}{2} 
        \begin{enumerate}
            \item \(\frac{1}{2} \times \frac{3}{4}\) \vspace{4cm}
            \item \(\frac{25}{5} \times \frac{6}{5}\) \vspace{4cm}
        \end{enumerate}
    \end{multicols}
    \vspace*{3cm}
    \item Additioner les fractions et les simplifier si besoin.
    \begin{multicols}{2} 
        \begin{enumerate}
            \item \(\frac{1}{2} + \frac{5}{2}\) \vspace{2cm}
            \item \(\frac{1}{3} + \frac{2}{5}\) \vspace{2cm}
        \end{enumerate}
    \end{multicols}
\end{enumerate}
\vspace{5cm}

% Quizz
\section{\textbf{Quizz}} 

\begin{table}[h!]
    \centering
    \renewcommand{\arraystretch}{3}
    \begin{tabular}{|c|c|c|c|} % 4colonnes
        \hline
        \textbf{Résoudre : \(3x(2x - 3) = 2x(3x+6)\)} & \( x = 0 \) & \( x = 1 \) & \( x = 2 \) \\ \hline
        \textbf{L'ordonnée à l'origine de la fonction \( f(x) = 2x-3 \) est... }   & 2  & -3  & \(\frac{2}{3}\)  \\ \hline
        \textbf{Le deuxième quartile est la...} & Moyenne & Somme des effectifs & Médiane \\ \hline
        \textbf{6400 équivaut à...} & \( 6,4 \times 10^3\) & \( 0,64 \times 10^5\) & \(6,4 \times 10^{-3}\) \\ \hline
    \end{tabular}
\caption{Quizz}
\end{table}

% Plan
\newpage
\section{\textbf{Géométrie}}
\begin{center}
    \begin{tikzpicture}
        \begin{axis}[
            axis lines = middle,
            xmin=-5, xmax=5, % Plage de x
            ymin=-5, ymax=5, % Plage de y
            xlabel = \(x\),
            ylabel = \(f(x)\),
            grid = both,
            width=10cm, % Largeur du graphique
            height=10cm, % Hauteur du graphique
            unit vector ratio = 1:1,
            xtick={-5,-4,-3,-2,-1,0,1,2,3,4,5}, % Positions des graduations principales sur l'axe x
            ytick={-5,-4,-3,-2,-1,0,1,2,3,4,5}, % Positions des graduations principales sur l'axe y
            major grid style={line width=0.8pt, draw=black!70}, % Grille principale plus épaisse et foncée
            minor grid style={line width=0.5pt, draw=black!50} % Grille secondaire plus fine et foncée
            ]
            \addplot[
                only marks, 
                mark=*,
                mark options={draw=black, fill=red}
                ] coordinates {(2,0)};
            \node[above right] at (axis cs:2,0) {\(C\)};
        \end{axis}
    \end{tikzpicture}
\end{center}

\section*{\textbf{Construction sur un plan}}  
\begin{itemize}
    \item Quelles sont les coordonnées du point C ? 
    \vspace{2cm}
    \item Construire sur le plan ci-dessus 3 points A et B avec les coordonnées 
    \[
    A(-1, 2) \quad \text{et} \quad B(4,-2)
    \]
    \vspace{0.5cm}
    \item Soit \( P(3, -1) \). Vérifier si \( P \) appartient à la droite \( (AB) \). 
    \vspace{2cm}
    \item Déterminer une équation de la droite passant par les points \( A \) et \( B \). 
    \vspace{2cm}
\end{itemize}

\newpage
\section{\textbf{Probabilité}} 
\textbf{Énoncé :} \\
Dans une urne, il y a trois boules vertes (\(V\)), six boules oranges (\(O\)) et deux boules bleues (\(B\)),
toutes indiscernables au toucher. \\
On tire sccessivement et sans remise de boules.

\begin{enumerate}[label=\textbf{Question \arabic*.}]
    \item Quelle est la probabilité de tirer une boule orange au premier tirage ?
    \vspace{0.5cm} 
    
    \begin{itemize}[label=]
        \item[A.] \( p(O) = \frac{6}{11} \) \vspace{0.5cm} 
        \item[B.] \( p(O) = \frac{3}{11} \) \vspace{0.5cm} 
        \item[C.] \( p(O) = \frac{2}{11} \) \vspace{0.5cm} 
        \item[D.] Autre réponse
    \end{itemize}
    \vspace{0.5cm} 
    
    \item Quelle est la probabilité qu la première soit bleue et la deuxième orange ?
    \vspace{0.5cm}
    
    \begin{itemize}[label=]
        \item[A.] \( p(B,O) = \frac{12}{21} \) \vspace{0.5cm} 
        \item[B.] \( p(B,O) = \frac{12}{121} \) \vspace{0.5cm} 
        \item[C.] \( p(B,O) = \frac{12}{110} \) \vspace{0.5cm} 
        \item[D.] Autre réponse
    \end{itemize}
    \vspace{0.5cm} 
    
    \item Quelle est la probabilité que la deuxième soit verte ?
    \vspace{0.5cm}
    \begin{itemize}[label=]
        \item[A.] \( p(V) = \frac{18}{10} \) \vspace{0.5cm} 
        \item[B.] \( p(V) = \frac{18}{10} \) \vspace{0.5cm} 
        \item[C.] \( p(V) = \frac{3}{11} \) \vspace{0.5cm} 
        \item[D.] Autre réponse
    \end{itemize}
\end{enumerate}


\end{document}